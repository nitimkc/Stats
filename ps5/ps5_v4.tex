\documentclass[a4paper, 11pt]{article}
\usepackage{covington}
\usepackage{amssymb}
\usepackage{amsmath}
\usepackage[catalan]{babel}
\usepackage{graphicx}
\usepackage{eurosym}
\usepackage{caption}
\usepackage{subcaption}
\usepackage{float}
\usepackage{bm}
\usepackage{layout}
\textheight=23.94cm 
\textwidth=17cm 
\topmargin=-1cm 
\oddsidemargin=-0.5cm 
 
\newcommand{\header}[4]{
	\begin{center}
		\rule{\linewidth}{0.5pt}
		
		{\small{#1}}
      
        \vspace{0.2in}
        
		{\large{#2}}
		
        \vspace{0.2in}
        
		{\small{#3}}
		
		\vspace{0.15in}
		
		{#4}
		
		\vspace{-0.1in}
		\rule{\linewidth}{0.6pt}
	\end{center}
}

\begin{document}
 
\header{\sc Barcelona Graduate School of Economics \hfill Master's Degree in Data Science}{\bf Statistical Modeling and Inference $-$ Problem Set \#5}{\sc Group 3: Niti Mishra $\cdot$ Miquel Torrens $\cdot$ B\'alint V\'an}{November 23\textsuperscript{rd}, 2015}
Solution to proposed exercises.\\
% EXERCISE 1
\newline \textbf{\underline{Exercise 1}}\\
\newline \underline{Part (a)}\\
\newline We define:
\begin{eqnarray}
\text{NdEF}(z | \theta, q) \equiv p(z | \theta, q) = \exp \lbrace q [z \theta - c(\theta)] + h(z,q) \rbrace
\end{eqnarray}
Let's start by the \textbf{normal} distribution:
\begin{eqnarray}
\mathcal{N}(t | \mu, q) &=& \frac{q^{1/2}}{\sqrt{2 \pi}} \exp \left\{ -\frac{q}{2} (t - \mu)^2 \right\}  \nonumber \\
&=& \frac{q^{1/2}}{\sqrt{2 \pi}} \exp \left\{ -\frac{1}{2} q (t^2 + \mu^2 - 2z\mu) \right\}  \nonumber \\
&=& \frac{q^{1/2}}{\sqrt{2 \pi}} \exp \left\{ q \left[t \mu -\frac{1}{2} \left( z^2 + \mu^2 \right) \right] \right\}  \nonumber \\
&=& \exp \left\{ q \left[ t \mu - \frac{1}{2} \mu^2 \right] + \log \left(  \frac{q^{1/2}}{\sqrt{2 \pi}} \right) - \frac{1}{2} q z^2  \right\}
\end{eqnarray}
Now using (1) we can see the following in (2):
\begin{eqnarray}
\mathcal{N}(t | \mu, q) = \exp \left\{ \underbrace{q}_{= q} \left[ \underbrace{t}_{=z} \underbrace{\mu}_{=\theta} - \underbrace{\frac{1}{2} \mu^2}_{=c(\theta)} \right]   + \underbrace{\log \left( \frac{q^{1/2}}{\sqrt{2 \pi}} \right) - \frac{1}{2} q z^2}_{=h(z,q)}  \right\}
\end{eqnarray}
Thus with the normal distribution: $z=t$, $q = q$, $\theta = \mu$ and $c(\theta) = \frac{1}{2} \mu^2 =  \frac{1}{2} \theta^2$.\\
\newline For the \textbf{Bernouilli} distribution:
\begin{eqnarray}
\text{Bern}(t | p) &=& p^t (1-p)^{1-t}  \nonumber \\
&=& \exp \left\{ t \log{p} + (1-t) \log{(1-p)} \right\} \nonumber \\
&=& \exp \left\{ t \log \left( \frac{p}{1-p} \right) + \log{(1-p)} \right\} \nonumber
\end{eqnarray}
Mimetizing the analysis in (3), we have that $z=t$, $q=1$, $\theta = \log \left( \frac{p}{1-p} \right)$ and that $c(\theta) = -\log{(1-p)} = \log(1 + e^\theta)$, because:
\begin{eqnarray}
\theta = \log \left( \frac{p}{1-p} \right) \Leftrightarrow p = \frac{e^\theta}{1 + e^\theta}.
\end{eqnarray}
For the \textbf{binomial} distribution:
\begin{eqnarray}
\text{Bin}(t | n,p) &=& {n \choose t} p^t (1-p)^{n-t}   \nonumber \\
&=& \exp \left\{ \log{n \choose t} + t \log p + (n-t) \log{(1-p)} \right\} \nonumber \\
&=& \exp \left\{ t \log \left( \frac{p}{1-p} \right) + n \log{(1-p)} + \log{n \choose t} \right\} \nonumber \\
&=& \exp \left\{ n \left[ \frac{t}{n} \log\left( \frac{p}{1-p} \right) + \log{(1-p)} \right] + \log{n \choose t} \right\} \nonumber
%%&=& \exp \left\{ \right\} \nonumber \\
\end{eqnarray}
Then for the binomial: $z = t/n$, $q=n$, $\theta = \log \left( \frac{p}{1-p} \right)$ and $c(\theta) = -\log{(1-p)} = \log(1 + e^\theta)$.\\
\newline Finally, for the \textbf{Poisson} distribution:
\begin{eqnarray}
\text{Pois}(t | \lambda) &=& \frac{\lambda^t e^{-\lambda}}{t!} \nonumber \\
&=& \exp \left\{ t \log \lambda - \lambda - \log t! \right\} \nonumber \\
&=& \exp \left\{ [t \log \lambda - \lambda] - \log t! \right\} \nonumber
\end{eqnarray}
And using (1) that means that for Poisson $z=t$, $q = 1$, $\theta = \log \lambda$ and $c(\theta) =\lambda = e^\theta$.\\
\newline \underline{Part (b)}\\
\newline To obtain the canonical links we apply the following formula\footnote{GLMs lecture, slide number 11.}:
\begin{eqnarray}
g(\mathbb{E}[z]) = \int \frac{1}{c''(\theta)} d\mathbb{E}[z]. \nonumber
\end{eqnarray}
For the \textbf{normal} distribution, where $\theta = \mu$:
\begin{eqnarray}
c(\theta) = \frac{1}{2} \theta^2. \nonumber
\end{eqnarray}
This means that $c'(\theta) = \theta$, $c''(\theta) = 1$ and $1/c''(\theta) = 1$. Then:
\begin{eqnarray}
g(\mu) = \int d\mu = \mu. \nonumber
\end{eqnarray}
For the \textbf{Bernouilli} and \textbf{binomial} distributions, which share $c(\theta)$, recall (4):
\begin{eqnarray}
c(\theta) = -\log{(1-p)} = -\log{\left( 1 - \frac{e^\theta}{1+e^\theta} \right)} = -\log{\left( \frac{1}{1+e^\theta} \right)} = \log{(1 + e^\theta)} \nonumber
\end{eqnarray}
From this we obtain:
\begin{eqnarray}
c'(\theta) &=& \frac{e^\theta}{1 + e^\theta} \nonumber \\
c''(\theta) &=& \frac{e^\theta}{(1 + e^\theta)^2}\nonumber \\
\frac{1}{c''(\theta)} &=& \frac{(1 + e^\theta)^2}{e^\theta} = \frac{\left(1 + \frac{p}{1-p}\right)^2}{\frac{p}{1-p}} \nonumber
\end{eqnarray}
And finally recover:
\begin{eqnarray}
g(p) = \int \frac{(1 + \frac{p}{1-p})^2}{\frac{p}{1-p}} dp = \log p - \log{(1-p)} = \log{\left(\frac{p}{1-p}\right)}. \nonumber
\end{eqnarray}
In the case of the \textbf{Poisson} distribution, we have that $\theta = \log \lambda$ and $c(\theta) = \lambda = e^{\theta}$, thus:
\begin{eqnarray}
c'(\theta) &=& e^{\theta} \nonumber \\
c''(\theta) &=& e^{\theta}  = \lambda \nonumber \\
\frac{1}{c''(\theta)} &=& \lambda^{-1} \nonumber
\end{eqnarray}
This implies:
\begin{eqnarray}
g(\lambda) = \int \lambda^{-1} d\lambda = \log \lambda. \nonumber
\end{eqnarray}
\newline \underline{Part (c)}\\
\newline From the log-likelihood function:
\begin{eqnarray}
\log \mathcal{L} \left( \text{NdEF}( t_n | \theta (\mathbf{x}_n, \mathbf{w}), q \gamma_n ) \right) = \sum_{n=1}^{N} \gamma_n q [ t_n \theta( \phi(\mathbf{x}_n)^T \mathbf{w}) - c(\theta(\phi(\mathbf{x}_n)^T \mathbf{w})) ] + \sum_{n=1}^{N} h(t_n, q \gamma_n) \nonumber
\end{eqnarray}
From the GLM lecture, slide 12:
\begin{eqnarray}
\frac{\partial \log \mathcal{L}}{\partial \mathbf{w}} &=& \sum_{n=1}^{N} \frac{\gamma_n}{V(y(\mathbf{x}_n, \mathbf{w}))} \frac{1}{g'(y(\mathbf{x}_n, \mathbf{w}))} q(t_n - y(\mathbf{x}_n, \mathbf{w})) \phi(\mathbf{x}_n)  \nonumber \\
&=& \sum_{n=1}^{N} \gamma_n q(t_n - y(\mathbf{x}_n, \mathbf{w})) \phi(\mathbf{x}_n) \nonumber \\
&=& q \sum_{n=1}^{N} \gamma_n t_n \phi(\mathbf{x}_n) - q \sum_{n=1}^{N} \gamma_n y(\mathbf{x}_n, \mathbf{w}) \phi(\mathbf{x}_n)   \nonumber
\end{eqnarray}
We can make the simplification because under the canonical form: $g'(y) = 1/V(y)$. Also under canonical form $\mathbb{E}[t_n] = \theta = g(y(\mathbf{x}_n, \mathbf{w})) = y(\mathbf{x}_n, \mathbf{w}) = \phi(\mathbf{x}_n)^T \mathbf{w}$. Then:
\begin{eqnarray}
\frac{\partial \log \mathcal{L}}{\partial \mathbf{w}} &=& q \sum_{n=1}^{N} \gamma_n t_n \phi(\mathbf{x}_n) - q \sum_{n=1}^{N} \gamma_n \phi(\mathbf{x}_n)^T \mathbf{w} \phi(\mathbf{x}_n)  \nonumber \\
\frac{\partial^2 \log \mathcal{L}}{\partial \mathbf{w}^2} &=& -q \sum_{n=1}^{N} \gamma_n \phi(\mathbf{x}_n)^T \phi(\mathbf{x}_n)
\end{eqnarray}
In matrix form write (5) as:
\begin{eqnarray}
-q \mathbf{\Phi}^T \mathbf{\Gamma} \mathbf{\Phi}, \nonumber
\end{eqnarray}
where $\mathbf{\Gamma} = \text{diag}(\gamma_n)$. We know that $\mathbf{\Phi}^T \mathbf{\Phi}$ would be positive semi-definite (proved in Exercise 1 of Problem Set 1), thus log-convex. All elements of $\mathbf{\Gamma}$ are non-negative and $q$ is as well, so this multiplication remains log-convex. Then the negative sign makes the log-likelihood log-concave.\\
% EXERCISE 2
\newpage
\textbf{\underline{Exercise 2}}\\
\newline \underline{Part (a)}\\
\newline From the slides on MLE regression\footnote{Slide 8}, we know that:
\begin{eqnarray}
-2 \log p(\mathbf{t} | \mathbf{X}, \mathbf{w}, q) = -N \log q + q (\mathbf{t} - \mathbf{\Phi w})^T (\mathbf{t} - \mathbf{\Phi w}) + c, \nonumber
\end{eqnarray}
where $c$ gathers all constant terms not dependent on $\mathbf{X}$. This yielded:
\begin{eqnarray}
q_{MLE} = \left( \frac{1}{N} \mathbf{e}^T \mathbf{e} \right)^{-1}. \nonumber
\end{eqnarray}
Then:
\begin{eqnarray}
-2 \log p(\mathbf{t} | \mathbf{X}, \mathbf{w}_{MLE}, q_{MLE}) &=& -N \log \left( \frac{1}{N} \mathbf{e}^T \mathbf{e} \right)^{-1} + \left( \frac{1}{N} \mathbf{e}^T \mathbf{e} \right)^{-1} (\mathbf{t} - \mathbf{\Phi} \mathbf{w}_{MLE})^T (\mathbf{t} - \mathbf{\Phi} \mathbf{w}_{MLE}) + c \nonumber \\
&=& N \log \mathbf{e}^T \mathbf{e} + N \left( \mathbf{e}^T \mathbf{e} \right)^{-1} \left( \mathbf{e}^T \mathbf{e} \right) + c \nonumber \\
&=& N \log \mathbf{e}^T \mathbf{e} + c. \nonumber
\end{eqnarray}
Thus proved.\\
\newline \underline{Part (b)}\\
\newline The problem in Part (a) also yielded:
\begin{eqnarray}
\mathbf{w}_{MLE} = \left( \mathbf{\Phi}^T \mathbf{\Phi} \right)^{-1} \mathbf{\Phi}^T \mathbf{t}. \nonumber
\end{eqnarray}
In the null model, $\mathbf{\Phi}$ is a column vector with $N$ components all of which are equal to 1. Consequently $\mathbf{\Phi}^T$ is a row vector also of length $N$. Then:
\begin{eqnarray}
\mathbf{\Phi}^T \mathbf{\Phi} = \sum_{i=1}^{N} 1 \times 1 = N. \nonumber
\end{eqnarray}
This implies:
\begin{eqnarray}
\left( \mathbf{\Phi}^T \mathbf{\Phi} \right)^{-1} = \frac{1}{N}. \nonumber
\end{eqnarray}
Also:
\begin{eqnarray}
\mathbf{\Phi}^T \mathbf{t} = \sum_{i=1}^{N} 1 \times t_i = \sum_{i=1}^{N} t_i. \nonumber
\end{eqnarray}
Finally:
\begin{eqnarray}
w_{0, MLE} = \left( \mathbf{\Phi}^T \mathbf{\Phi} \right)^{-1} \mathbf{\Phi}^T \mathbf{t} = \frac{1}{N} \sum_{i=1}^{N} t_i = \bar{t}. \nonumber
\end{eqnarray}
Hence proved.\\
\newline \underline{Part (c)}\\
\newline Using the results in Exercise 2.1:
\begin{eqnarray}
D_0 &=& -2 \log p(\mathbf{t} | \mathbf{X}_{m_0} \mathbf{w}_{m_0, MLE}, q_{m_0, MLE})  \nonumber \\
&=& N \log (\mathbf{t} - \bar{t})^T (\mathbf{t} - \bar{t}) + c \nonumber \\
&=& N \log q^{-1} + c \nonumber \\
&=& - N \log q + c \nonumber \\
\nonumber \\
D_1 &=& -2 \log p(\mathbf{t} | \mathbf{X}_{m_1} \mathbf{w}_{m_1, MLE}, q_{m_1, MLE})  \nonumber \\
&=& N \log (\mathbf{e}^T \mathbf{e}) + c \nonumber
\end{eqnarray}
Also note that:
\begin{eqnarray}
R^2 = 1 - \frac{\text{var}(\mathbf{t} | \mathbf{x}) }{\text{var}(\mathbf{t})} = 1 - \frac{(\mathbf{t} - \hat{\mathbf{t}})^T (\mathbf{t} - \hat{\mathbf{t}})}{(\mathbf{t} - \bar{t})^T (\mathbf{t} - \bar{t})} = 1 - \frac{\mathbf{e}^T \mathbf{e}}{q^{-1}} = 1 - q \mathbf{e}^T \mathbf{e}  \nonumber
\end{eqnarray}
Now we calculate:
\begin{eqnarray}
D_0 - D_1 &=& (-N \log q + c) - (N \log \mathbf{e}^T \mathbf{e} + c) \nonumber \\
&=& -N \log q - N \log \mathbf{e}^T \mathbf{e} \nonumber \\ 
&=& -N  \log q \mathbf{e}^T \mathbf{e} \nonumber \\ 
&=& -N  \log \left(1 - (1 - q \mathbf{e}^T \mathbf{e})\right) \nonumber \\ 
&=& -N  \log \left(1 - R^2 \right). \nonumber
\end{eqnarray}
This finishes the proof.

\end{document}
