\documentclass[a4paper, 11pt]{article}
\usepackage{covington}
\usepackage{amssymb}
\usepackage{amsmath}
\usepackage[catalan]{babel}
\usepackage{graphicx}
\usepackage{eurosym}
\textheight=23.94cm 
\textwidth=17cm 
\topmargin=-1cm 
\oddsidemargin=-0.5cm 
 
\newcommand{\header}[4]{
	\begin{center}
		\rule{\linewidth}{0.5pt}
		
		{\small{#1}}
      
        \vspace{0.2in}
        
		{\large{#2}}
		
        \vspace{0.2in}
        
		{\small{#3}}
		
		\vspace{0.15in}
		
		{#4}
		
		\vspace{-0.1in}
		\rule{\linewidth}{0.6pt}
	\end{center}
}

\begin{document}
 
\header{\sc Barcelona Graduate School of Economics \hfill Master's Degree in Data Science}{\bf Statistical Modeling and Inference $-$ Problem Set \#2}{\sc Niti Mishra $\cdot$ Miquel Torrens $\cdot$ B\'alint V\'an}{October 18\textsuperscript{th}, 2015}
Solution to proposed exercises.\\
% PROBLEM SET 2 (PART 1)
% EXERCISE 1
\newline \textbf{\underline{Exercise 1}}\\
\newline We need to solve:
\begin{eqnarray}
\max_{\mathbf{w}} -2 \log p(\mathbf{w | t, X}) &=&  -2q \mathbf{t}^T \mathbf{\Phi w} + q \mathbf{w}^T \mathbf{\Phi}^T \mathbf{\Phi w} + (\mathbf{w - \mu})^T \mathbf{D} (\mathbf{w - \mu}) + C \nonumber \\
&=&  -2q \mathbf{t}^T \mathbf{\Phi w} + q \mathbf{w}^T \mathbf{\Phi}^T \mathbf{\Phi w} + \mathbf{w}^T \mathbf{D w} - 2 \mathbf{w}^T \mathbf{D} \mathbf{\mu} + \mathbf{\mu}^T \mathbf{D \mu} + C \nonumber
\end{eqnarray}
The value $C$ includes all constant terms not depending on $\mathbf{w}$. Now we maximize with respect to $\mathbf{w}$ and set to zero:
\begin{eqnarray}
-2q \mathbf{t}^T \mathbf{\Phi} + q \mathbf{w}^T \left(\mathbf{\Phi}^T \mathbf{\Phi} + \left( \mathbf{\Phi}^T \mathbf{\Phi} \right)^T \right) + \mathbf{w}^T \left( \mathbf{D} + \mathbf{D}^T \right) - 2 \left( \mathbf{D} \mathbf{\mu} \right)^T = 0 \nonumber
\end{eqnarray}
During the derivation we will recurrently use two properties: $\mathbf{D} = \mathbf{D}^T$, as it is symmetric by construction, and $\left( \mathbf{\Phi}^T \mathbf{\Phi} \right)^T = \mathbf{\Phi}^T \mathbf{\Phi}$, which is a straightforward calculation. We just need to rearrange terms to reach the normal equations:
\begin{eqnarray}
2 \mathbf{w}^T\mathbf{D} -2 \mathbf{\mu}^T \mathbf{D}^T -2 q \mathbf{t}^T \mathbf{\Phi} + 2q\mathbf{w}^T \mathbf{\Phi}^T \mathbf{\Phi} &=& 0 \nonumber \\
\mathbf{w}^T \left( \mathbf{D} + q \mathbf{\Phi}^T \mathbf{\Phi} \right) &=& q\mathbf{t}^T \mathbf{\Phi} + \left( \mathbf{D\mu} \right)^T  \nonumber \\
\left( \mathbf{D} + q \mathbf{\Phi}^T \mathbf{\Phi} \right)^T \mathbf{w} &=& q\mathbf{\Phi}^T \mathbf{t} + \left( \mathbf{D\mu} \right) \nonumber
\end{eqnarray}
To finally obtain the normal equations:
\begin{eqnarray}
\left( \mathbf{D} + q \mathbf{\Phi}^T \mathbf{\Phi} \right) \mathbf{w} = q\mathbf{\Phi}^T \mathbf{t} + \mathbf{D\mu} \nonumber
\end{eqnarray}
Hence proved.\\
% EXERCISE 2
\newline \textbf{\underline{Exercise 2}}\\
% PROBLEM SET 2 (PART 2)
% EXERCISE 3
\newline \textbf{\underline{Exercise 3}}\\
% EXERCISE 4
\newline \textbf{\underline{Exercise 4}}\\

\end{document}