\documentclass[a4paper, 11pt]{article}
\usepackage{covington}
\usepackage{amssymb}
\usepackage{amsmath}
\usepackage[catalan]{babel}
\usepackage{graphicx}
\usepackage{eurosym}
\usepackage{caption}
\usepackage{subcaption}
\usepackage{float}
\usepackage{bm}
\usepackage{layout}
\textheight=23.94cm 
\textwidth=17cm 
\topmargin=-1cm 
\oddsidemargin=-0.5cm 
 
\newcommand{\header}[4]{
	\begin{center}
		\rule{\linewidth}{0.5pt}
		
		{\small{#1}}
      
        \vspace{0.2in}
        
		{\large{#2}}
		
        \vspace{0.2in}
        
		{\small{#3}}
		
		\vspace{0.15in}
		
		{#4}
		
		\vspace{-0.1in}
		\rule{\linewidth}{0.6pt}
	\end{center}
}

\begin{document}
 
\header{\sc Barcelona Graduate School of Economics \hfill Master's Degree in Data Science}{\bf Statistical Modeling and Inference $-$ Problem Set \#4}{\sc Niti Mishra $\cdot$ Miquel Torrens $\cdot$ B\'alint V\'an}{November 9\textsuperscript{th}, 2015}
Solution to proposed exercises.\\
% EXERCISE 1
\newline \textbf{\underline{Exercise 1}}\\
\newline We need to show that $y(\mu + \sigma)$ is a point less than one standard deviation away from the mean of the marginal distribution of $t$, that is:
\begin{eqnarray}
y(\mu + \sigma) \leq \bar{t} + \sqrt{\mathbb{V}[t]} \nonumber
\end{eqnarray}
Given that $x \sim \mathcal{N}(\mu, \sigma^2)$ and $\varepsilon \sim \mathcal{N}(0, \tau^2)$ are assumed to be uncorrelated:
\begin{eqnarray}
\mathbb{E}[t] &=& \mathbb{E}[x + \varepsilon] = \mathbb{E}[x] + \mathbb{E}[\varepsilon] = \mu = \bar{t} \nonumber \\
\mathbb{V}[t] &=& \mathbb{V}[x + \varepsilon] = \mathbb{V}[x] + \mathbb{V}[\varepsilon] = \sigma^2 + \tau^2 \nonumber
\end{eqnarray}
We see that $\mu = \bar{t}$ because given the distribution of its components $t$ is a normally (and thus symmetrically) distributed around its mean, and has the same expected value as $x$. On the other hand:
\begin{eqnarray}
y(\mu + \sigma) = \mathbb{E}[t | x = \mu + \sigma] = \mu + \sigma \nonumber
\end{eqnarray}
And so we would need to show that:
\begin{eqnarray}
y(\mu + \sigma) &\leq& \mu + \sqrt{\mathbb{V}[t]} \nonumber \\
\mu + \sigma &\leq& \mu + \sqrt{\sigma^2 + \tau^2} \nonumber \\
\sigma^2 &\leq& \sigma^2 + \tau^2  \nonumber \\
\tau^2 &\geq& 0 \nonumber
\end{eqnarray}
We know that $\tau^2 \geq 0$ is indeed non-negative, thus proved.\\
% EXERCISE 2
\newpage
\textbf{\underline{Exercise 2}}\\
\newline \underline{Part (a)}\\
\newline Answer.\\
\newline \underline{Part (b)}\\
\newline Answer.\\
\newline \underline{Part (c)}\\
\newline Answer.\\
\newline \underline{Part (d)}\\
\newline Answer.
% EXERCISE 3
\newpage
\textbf{\underline{Exercise 3}}\\
\newline \underline{Part (a)}\\
\newline Answer.\\
\newline \underline{Part (b)}\\
\newline Answer.\\
\newline \underline{Part (c)}\\
\newline Answer.
% EXERCISE 4
\newpage
\textbf{\underline{Exercise 4}}\\
\newline Answer.

\end{document}





