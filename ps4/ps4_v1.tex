\documentclass[a4paper, 11pt]{article}
\usepackage{covington}
\usepackage{amssymb}
\usepackage{amsmath}
\usepackage[catalan]{babel}
\usepackage{graphicx}
\usepackage{eurosym}
\usepackage{caption}
\usepackage{subcaption}
\usepackage{float}
\usepackage{bm}
\usepackage{layout}
\textheight=23.94cm 
\textwidth=17cm 
\topmargin=-1cm 
\oddsidemargin=-0.5cm 
 
\newcommand{\header}[4]{
	\begin{center}
		\rule{\linewidth}{0.5pt}
		
		{\small{#1}}
      
        \vspace{0.2in}
        
		{\large{#2}}
		
        \vspace{0.2in}
        
		{\small{#3}}
		
		\vspace{0.15in}
		
		{#4}
		
		\vspace{-0.1in}
		\rule{\linewidth}{0.6pt}
	\end{center}
}

\begin{document}
 
\header{\sc Barcelona Graduate School of Economics \hfill Master's Degree in Data Science}{\bf Statistical Modeling and Inference $-$ Problem Set \#4}{\sc Niti Mishra $\cdot$ Miquel Torrens $\cdot$ B\'alint V\'an}{November 9\textsuperscript{th}, 2015}
Solution to proposed exercises.\\
% EXERCISE 1
\newline \textbf{\underline{Exercise 1}}\\
\newline We need to show that $y(\mu + \sigma)$ is a point less than one standard deviation away from the mean of the marginal distribution of $t$, that is:
\begin{eqnarray}
y(\mu + \sigma) \leq \bar{t} + \sqrt{\mathbb{V}[t]} \nonumber
\end{eqnarray}
Given that $x \sim \mathcal{N}(\mu, \sigma^2)$ and $\varepsilon \sim \mathcal{N}(0, \tau^2)$ are assumed to be uncorrelated:
\begin{eqnarray}
\mathbb{E}[t] &=& \mathbb{E}[x + \varepsilon] = \mathbb{E}[x] + \mathbb{E}[\varepsilon] = \mu = \bar{t} \nonumber \\
\mathbb{V}[t] &=& \mathbb{V}[x + \varepsilon] = \mathbb{V}[x] + \mathbb{V}[\varepsilon] = \sigma^2 + \tau^2 \nonumber
\end{eqnarray}
We see that $\mu = \bar{t}$ because given the distribution of its components $t$ is a normally (and thus symmetrically) distributed around its mean, and has the same expected value as $x$. On the other hand:
\begin{eqnarray}
y(\mu + \sigma) = \mathbb{E}[t | x = \mu + \sigma] = \mu + \sigma \nonumber
\end{eqnarray}
And so we would need to show that:
\begin{eqnarray}
y(\mu + \sigma) &\leq& \mu + \sqrt{\mathbb{V}[t]} \nonumber \\
\mu + \sigma &\leq& \mu + \sqrt{\sigma^2 + \tau^2} \nonumber \\
\sigma^2 &\leq& \sigma^2 + \tau^2  \nonumber \\
\tau^2 &\geq& 0 \nonumber
\end{eqnarray}
We know that $\tau^2 \geq 0$ is indeed non-negative, thus proved.\\
% EXERCISE 2
\newpage
\textbf{\underline{Exercise 2}}\\
\newline \underline{Part (a)}\\
\newline We perform the following transformations to the raw data:
\begin{itemize}
\item Unify the fields \texttt{EARN1} and \texttt{EARN2} in one single field, adding an extra field named \texttt{INEXACT} that captures whether we use the precise answer from \texttt{EARN1} or the approximated answer in \texttt{EARN2}.
\item The resulting variable is called \texttt{EARNT} and is measured in thousands of dollars (e.g. \$10,000 has $\texttt{EARNT} = 10$).
\item Convert the field \texttt{HEIGHT} to total amount of inches and name it \texttt{HEIGHT\_I}.
\item Reescale the variable \texttt{SEX} to variable \texttt{MEN}, which takes value 1 if the individual is a man and 0 otherwise.
\item We suppress individuals with a reported weight greater than 500 (some have 990+ values for answer-codification reasons).
\item We suppress individuals with a reported height greater than 8 feet (some have 990+ values for answer-codification reasons).
\item We cut off the individual with highest income, as he reports an income that doubles the second highest income.
\end{itemize}
\underline{Part (b)}\\
\newline The regression run is the following:\\
\includegraphics[scale=0.7]{reg1.png}
\newline The resulting parameter for \texttt{HEIGHT\_I} is imperceptibly sensitive to the inclusion of the control variable for exactness of income, although including it does boost the $R^2$ considerably and reduces the residual standard error.\\
\newline The transformation needed to interpret the intercept as average earnings for people with average height is substracting the mean fro the \texttt{HEIGHT\_I} variable. If we do so the resulting intercept is the following:\\
\includegraphics[scale=0.7]{reg2.png}
\newline This states that a person with average height will earn on average an income of  \$28,011 approximately.\\
\newline \underline{Part (c)}\\
\newline We have run the following models:\\
\includegraphics[scale=0.5]{reg3.png}
\newline In the models that use logarithms we only use the observations with positive earnings.\\
\newline We choose model \texttt{m13} for several reasons. First, the residual standard error is the smallest between within log-linear models; second, it has a relatively high $R^2$ and, third, the regressors are all significant and have an intuitive sign. The results are the following:\\
\includegraphics[scale=0.7]{reg4.png}
\newline \\
\newline \underline{Part (d)}\\
\newline The results are only for strictly positive income individuals. The interpretation of the coefficients is the following:
\begin{itemize}
\item The intercept says that a woman of average height and weight (who would report the exact amount of salary) is expected to earn $1,000 \times \exp{(2.5956)} = \$13,405$.
\item The coefficient for \texttt{HEIGHT\_I\_ST} says that exceeding the average height by one standard deviation increases the expected income by 8.1\%. Exceeding by two standard deviations would increase the expected salary by 16.2\%, and so on. In the case of men, this is diminished by the coefficient of \texttt{HEIGHT\_I\_ST:MEN}, which sets the total increase for men to 5.7\% when one standard deviation away, although this coefficient is not significant.
\item The coefficient for \texttt{WEIGHT\_ST} says that exceeding the average weight by one standard deviation decreases the expected income by 7.6\%. In the case of men, this is actually overturned by the coefficient on \texttt{WEIGHT\_ST:MEN}, and the aggregated effect is of +10.9\% on income, so the data shows a negative effect on women and positive on men.
\item The coefficient on \texttt{MEN} suggests that an man of average height and weight earns on average 27.3\% more than a woman of average height and weight.
\item The coefficient on \texttt{INEXACT} means that a woman who reports inexact answers is expected to earn 69.9\% more. In the case of men, we add the value of the interaction term to expect an increase of 31.7\% for the average man.
\end{itemize}
% EXERCISE 3
\newpage
\textbf{\underline{Exercise 3}}\\
\newline \underline{Part (a)}\\
\newline We generate random draws and plot the result:
\begin{center}
\includegraphics[scale=0.6]{plot_ex3dot1.png}
\end{center}
\underline{Part (b)}\\
\newline We take the sample ratio of the simulations we have drawn. We take the quantile of this sample ratio and keep the corresponding quantiles.\\
\newline For the 50\% case, we take quantiles 25\% and 75\%, leaving 50\% of the sample inside (50\% confidence interval). The interval is $(108.7, 313.7)$\\
\newline For the 50\% case, we take quantiles 2.5\% and 97.5\%, leaving 95\% of the sample inside (95\% confidence interval). The interval is $(-57.3, 677.9)$.\\
\newline \underline{Part (b)}\\
\newline The plot is the following:
\begin{center}
\includegraphics[scale=0.6]{plot_ex3dot3.png}
\end{center}
The intervals in this case are the following:
\begin{itemize}
\item 50\% confidence interval: $(71.2, 326.9)$.
\item 95\% confidence interval: $(-1386.8, 2122.8)$.
\end{itemize}
% EXERCISE 4
\newpage
\textbf{\underline{Exercise 4}}\\
\newline Answer.

\end{document}





